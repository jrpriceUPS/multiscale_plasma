\documentclass{article}
\usepackage{setspace}
\usepackage{amsfonts}
\usepackage{graphicx}
\usepackage{amsmath}
\graphicspath{{./figures/}}

\title{Computing $\tau_{kl}(x)$ in HMM in 1-, 2-, and $n$-Species Systems}
\author{Jake Price}

\begin{document}
\maketitle



\section{Assumptions and Definitions}

Let the Klimontovich distribution for a system with $n$ species and $N$ total ions be defined as:
\begin{equation}
\mathcal{N}(\mathbf{r},\mathbf{v},\{\mathbf{r}_i\}_{i=1}^N,\{\mathbf{v}_i\}_{i=1}^N)=\sum_i \delta(\mathbf{r}-\mathbf{r}_i(t))\delta(\mathbf{v}-\mathbf{v}_i(t)).
\end{equation}For species $k$, let the set of indices corresponding to ions in that species be given by $S_k=\{i\;|\;$particle $i$ is a member of species $k\}$. Then the Klimontovich distribution for species $k$ is given by
\begin{equation}
\mathcal{N}_k(\mathbf{r},\mathbf{v},\{\mathbf{r}_i\}_{i=1}^N,\{\mathbf{v}_i\}_{i=1}^N)=\sum_{i\in S_k}\delta(\mathbf{r}-\mathbf{r}_i(t))\delta(\mathbf{v}-\mathbf{v}_i(t)).
\end{equation}Note that, by definition $\mathcal{N}=\sum_k \mathcal{N}_k$. Note also that the expected value of these distributions are $f_k$ and $f$ respectively (again by definition).

Suppose now that, over the macroscopic domain, these distributions evolve according to the BGK equation with periodic boundary conditions and Yukawa potential:
\begin{equation}
\begin{split}
&\frac{\partial f_k}{\partial t}+\mathbf{v}\cdot\nabla_\mathbf{r}f_k-\frac{Z_ke}{m_k}\nabla_\mathbf{r}\phi(\mathbf{r},t)\cdot\nabla_\mathbf{v}f_k=\sum_l\frac{f_{kl}^{eq}-f_k}{\tau_{kl}}\\
&\left(\bigtriangleup_\mathbf{r}-\frac{1}{\lambda^2}\right)\phi(\mathbf{r},t)=-\frac{1}{\epsilon_0}\rho(\mathbf{r}',t)\\
&f_k(\mathbf{r},\mathbf{v},0)=f_{k}(\mathbf{r},\mathbf{v})\\
&\left.f_k(\mathbf{r},\mathbf{v},t)\right|_{x=0}=\left.f_{k}(\mathbf{r},\mathbf{v},t)\right|_{x=L_x},\;\;\;\;
\left.\phi(\mathbf{r},\mathbf{v},t)\right|_{x=0}=\left.\phi(\mathbf{r},\mathbf{v},t)\right|_{x=L_x}\\
&\left.f_k(\mathbf{r},\mathbf{v},t)\right|_{y=0}=\left.f_{k}(\mathbf{r},\mathbf{v},t)\right|_{y=L_y},\;\;\;\;
\left.\phi(\mathbf{r},\mathbf{v},t)\right|_{y=0}=\left.\phi(\mathbf{r},\mathbf{v},t)\right|_{y=L_y}\\
&\left.f_k(\mathbf{r},\mathbf{v},t)\right|_{z=0}=\left.f_{k}(\mathbf{r},\mathbf{v},t)\right|_{z=L_z},\;\;\;\;
\left.\phi(\mathbf{r},\mathbf{v},t)\right|_{z=0}=\left.\phi(\mathbf{r},\mathbf{v},t)\right|_{z=L_z}.
\end{split}
\label{eq:BGK}
\end{equation}
The quantities $\tau_{kl}$ are unknown. They are allowed to vary in space and time, but not in velocity. They are also not necessarily symmetric ($\tau_{kl}(\mathbf{r},t)\neq \tau_{lk}(\mathbf{r},t)$).

\section{One Species}

Consider computing $\tau(\mathbf{r},t)$ for a system with only one species. We will introduce the entropy of the system, which we define as:
\begin{equation}
\mathcal{H}(t)=\iint f\log(f)\,d\mathbf{r}\,d\mathbf{v}.
\end{equation}Our goal is to select $\tau(\mathbf{r},t)$ such that, at position $\mathbf{r}$ and time $t$ the rate of change of the entropy is the same in an MD simulation and in the BGK simulation. Therefore, we take the time derivative and use the BGK equation:
\begin{align*}
\frac{\partial \mathcal{H}}{\partial t}=&\iint \frac{\partial f}{\partial t}[\log(f)+1]\,d\mathbf{r}\,d\mathbf{v}\\
=&\iint \frac{f^{eq}-f}{\tau}[\log(f)+1]\,d\mathbf{r}\,d\mathbf{v}-\iint \mathbf{v}\cdot \nabla_\mathbf{r}f[\log(f)+1]\,d\mathbf{r}\,d\mathbf{v}\\
&+\frac{Ze}{m}\nabla_\mathbf{r}\phi\cdot\nabla_\mathbf{v}f[\log(f)+1]\,d\mathbf{r}\,d\mathbf{v}
\end{align*}





\end{document}